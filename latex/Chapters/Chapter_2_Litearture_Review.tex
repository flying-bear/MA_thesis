% Chapter 2

\chapter{Literature Review} % Main chapter title
\label{chap:2:review} 

In this section, I provide the summarized overview of a number of papers dedicated to psychosis detection. The section is structured around the linguistic types of metrics outlined in the introduction, and papers may appear several times if they use a variety of metrics of different types. The papers were searched for with keywords relating to psychosis\footnote{``Psychosis'', ``schizophrenia'', and ``thought disorder''.} and NLP\footnote{``Automated language'', ``embedding'',  ``model'', ``NLP''.}. The citations of the retrieved articles were used to expand the search. The articles concerning purely manual linguistic metrics or non-psychotic conditions were excluded from analysis, yielding a total of 61 papers and conference materials. The information about the patient populations of the papers reviewed can be found in appendix \ref{appendix:datasets}.

Herein, I will pay most attention to the metrics used in multiple papers. The sections cover each of the groups outlined in the introduction: lexical (\ref{sec:review:lexical}), syntactic (\ref{sec:review:syntactic}), and semantic methods, including graph-based (\ref{sec:review:graph}) and LM-based ones (\ref{sec:review:LM}). I also provide a cross-group comparison of the methods (\ref{sec:review:cross-group}) and a short summary of the trends observed (\ref{sec:review:summary}). The results are summarized in table format in appendix \ref{appendix:methods}.

%----------------------------------------------------------------------------------------
%	SECTION 1
%----------------------------------------------------------------------------------------

\section{Lexical Methods}
\label{sec:review:lexical}

Almost all studies that report the difference in word count report lower \textit{verbosity} in the patient group as compared to control\footnote{\cite{mota2014graph, iter2018automatic, willits2018evidence, dore2019quantification, just2019coherence, just2020modeling, panicheva2020corpus, morgan2021natural, spencer2021lower, voppel2021quantified, liang2022widespread, parola2022speech}} or association with negative symptom severity \citep{morgan2021natural, minor2023automated}, with some studies finding no difference in verbosity\footnote{\citep{mota2012speech, gupta2018automated, tang2021natural, alonso2022language, alonso2022progressive, schneider2023syntactic, nettekoven2023semantic}}. 

Of the various measures of \textit{lexical diversity}, the most frequently used one is type-token ration (\textit{TTR}), defined as the number of unique words divided by the total number of words. While many studies use this metric or its variations \footnote{\cite{rosenstein2015language, willits2018evidence, kramov2020evaluating, hitczenko2021understanding, liang2022widespread, aich2022towards, ziv2022morphological, jeong2023exploring, minor2023automated, schneider2023syntactic, tang2023latent}.}, a large portion of them only uses it as a feature for a classifier or latent analysis. In cases, where TTR is analyzed on its own, it is usually reported to be lower in the patient group \citep{willits2018evidence, aich2022towards, minor2023automated}, though some report no significant differences \citep{hitczenko2021understanding, jeong2023exploring, schneider2023syntactic}, or higher \textit{type} and \textit{lemma token ratio} \citep{ziv2022morphological}. The \textit{number of unique words} is also reported to be lower in the patient population by some \citep{willits2018evidence} but not others \citep{schneider2023syntactic}. Additional lexical diversity measures such as \textit{Honoré statistics} are reported to be correlated with symptom severity \citep{jeong2023exploring}. 

Another major group of lexical metrics is \textit{sentiment analysis} metrics. Many researchers rely on pre-defined emotion word dictionaries provided by such tools as LIWC\footnote{\cite{mitchell2015quantifying, vail2018toward, girard2022computational, mota2022happy, minor2023automated}}, Senti-WordNet, or NRC Word-Emotion Lexicon \citep{aich2022towards}. \citet{mitchell2015quantifying} report higher negative emotion words and lower positive emotion words in the schizophrenia group and \citet{aich2022towards} report lower trust words use. \citet{vail2018toward} and \citet{girard2022computational} report negative emotion words correlating withe negative symptom severity, and \citet{minor2023automated} report lower positive emotion word use and higher negative emotion word use being associated with symptom severity. Alternatively, trained sentiment analysis models can be utilized \citep{tang2022clinical, tang2023latent} to produce classifier features.

% Higher percentages indicate more frequent word use within a category. The LIWC has been used to measure verbal emotional expression (Kahn et al., 2007; Minor et al., 2015a) and to assess word use in schizophrenia (Buck et al., 2015a; Cohen et al., 2009; Fineberg et al., 2015; St-Hilaire et al., 2008). 

Additionally, different ways of quantifying \textit{atypical word use} are reported to differentiate between controls and patients, including the use of neologisms \citep{just2019coherence, just2020modeling} and out-of-vocabulary words \citep{just2020modeling}, and the use of foreign words is reported to be correlated with symptom severity \citep{jeong2023exploring}.

Finally, many researchers rely on such tools as LWIC, TAACO, or Coh-Metrix to provide them with a wide range of lexical and syntactic features, including various repetition and overlap patterns, cohesion or connective word categories, POS tags, and topic analysis\footnote{\cite{mitchell2015quantifying, gupta2018automated, vail2018toward, willits2018evidence, just2020modeling, mota2022happy, girard2022computational, liang2022widespread, minor2023automated}}. Some researchers also utilize latent content analysis to explore prevalent topics, which is especially useful on heterogeneous texts \citep{mitchell2015quantifying, rezaii2019machine}. While these tools are popular and often prove useful for psychosis prediction, different researchers focus on different output features and topics, some also only using the features as input to classifiers or latent models, rendering direct comparisons uninformative. 

The results for lexical methods are summarized in table \ref{tab:method:lexical} in appendix. In the present work, I intend to assess the \textcolor{red}{verbosity and lexical diversity}.  

%----------------------------------------------------------------------------------------
%	SECTION 2
%----------------------------------------------------------------------------------------

\section{Syntactic Methods}
\label{sec:review:syntactic}

Analyzing the variation in distribution of \textit{part-of-speech} (POS) is the most popular syntactic method as it is quite easy to perform and quantify. The most consistently reported change is the lower use of determiners and wh-words in the patient group \citep{bedi2015automated, corcoran2018prediction, sarzynska2021detecting, tang2021natural}, though with some reporting no group difference on CHR population \citep{bilgrami2022construct} or no correlation with symptoms \citep{corcoran2018prediction, bilgrami2022construct}. \citet{mitchell2015quantifying} report higher article use rather than lower. Some studies report lower adjective and adverb use \citep{corcoran2018prediction, tang2021natural, ziv2022morphological} and reduced typicality in the use of adjectives and adverbs \citep{bar2019semantic}, while others report higher adjective use in at risk populations \citep{argolo2023burnishing}. The patients were also reported to show lower verb and past tense use \citep{ziv2022morphological} though some report the opposite results \citep{mitchell2015quantifying} or no group differences in verb use \citep{tang2021natural, argolo2023burnishing}. Additionally, higher subordinating conjunction use \citep{silva2022syntactic}, lower possessive pronoun use \citep{corcoran2018prediction} and higher 1st person word use \citep{ziv2022morphological} or negative correlation of pronoun use with symptom severity \citep{jeong2023exploring} were reported for patient populations. Additionally, several studies use POS tags as features in predictive models or latent analysis \citep{bedi2015automated, sarzynska2021detecting, tang2022clinical, tang2023latent}. The only study to analyze POS tags and report no differences focused on clinical high risk population, rather than patients exhibiting symptoms \citep{haas2020linking}. It is important to remark that two different major tagging schemes were used in the studies, namely, \href{https://universaldependencies.org/u/pos/}{universal POS tagging}\footnote{https://universaldependencies.org/u/pos/} and \href{https://www.ling.upenn.edu/courses/Fall_2003/ling001/penn_treebank_pos.html}{Penn Treebank POS tagging}\footnote{https://www.ling.upenn.edu/courses/Fall\_2003/ling001/penn\_treebank\_pos.html} schemes, and some languages lack some of the POS categories commonly used for English, rendering some of the results incomparable.

Secondly, several studies assess the use more ambiguous pronouns and cataphora \citep{iter2018automatic, morgan2021natural, nettekoven2023semantic}, which is either detected automatically, using a coreference tool, such as e2e\_coref \citep{lee2017end}, or manually \citep{just2020modeling}. While some report higher referential failures in the patient population \citep{iter2018automatic, just2020modeling}, others report no difference in the referential failure rates \citep{morgan2021natural}.

Several works propose various measures of syntactic complexity. Some measures are based on the use of different syntactic roles, reporting lower use of passive nominal subjects, clausal negation, and prepositional complements, but higher use of simple nominal subjects in the patient population \citep{silva2022syntactic}. Other measure syntactic complexity based on clause types, reporting lower use of coordinated clauses, lower index of general syntactic complexity, and higher rates of simple sentences \citep{schneider2023syntactic} as well as correlating lower syntactic complexity (measured as reduced number of T-units per sentence) with symptom severity \citep{jeong2023exploring}. Some studies rely on freely available tools, such as \href{https://www.linguisticanalysistools.org/taassc.html}{TAASSC}\footnote{ Tool for the Automatic Analysis of Syntactic Sophistication and Complexity (TAASSC), available at https://www.linguisticanalysistools.org/taassc.html, \citep{kyle2016measuring}}, to assess syntactic complexity \citep{liang2022widespread, silva2022syntactic}.

Finally, some rely on sentence, clause, or phrase length and counts as a metric. Generally, decreased sentence length is reported in the patient population \footnote{\citep{iter2018automatic, morgan2021natural, spencer2021lower, tang2021natural, bilgrami2022construct, silva2022syntactic, nettekoven2023semantic, schneider2023syntactic}}, though some report no differences between patients with FEP \citep{liang2022widespread} or CHR \citep{gupta2018automated, haas2020linking} and controls. Additionally, some report lower sentence length correlating with positive \citep{liebenthal2022linguistic} and negative \citep{bilgrami2022construct} symptoms, as well as social functioning \citep{silva2022syntactic}. Maximal sentence length was reported to correlate with poverty of speech as assessed by TALD \citep{xu2020centroid}. Some also report lower clause or T-unit length in patient population \citep{silva2022syntactic} or in the patient sub-population with more severely affected brain \citep{liang2022widespread} with no difference between the patient group and control. Several studies successfully use utterance length \citep{tang2023latent} or maximal phrase length \citep{bedi2015automated} as a feature in latent analysis or classifiers. Interestingly, the number of sentences is rarely used, and the results are contradictory, with some reporting lower count \citep{iter2018automatic} or correlation with symptoms \citep{jeong2023exploring} while others find no group differences \citep{gupta2018automated, tang2021natural, schneider2023syntactic} or higher sentence counts \citep{morgan2021natural, nettekoven2023semantic}. This might indicate that lower verbosity overall is a result of shorter simpler sentences, rather than lower sentence counts.

The results for syntactic methods are summarized in table \ref{tab:method:syntactic} in appendix. Among syntactic methods, I intend to test, firstly, major \textcolor{red}{parts of speech paying particular attention to determiners, adjectives, and pronouns}. Secondly, I intend to assess \textcolor{red}{the length and number of sentences (or utterances)} as it is one of the frequently used metrics and a metric that may influence other metrics.

%----------------------------------------------------------------------------------------
%	SECTION 3
%----------------------------------------------------------------------------------------
\section{Graph-Based Methods}
\label{sec:review:graph}

% Graph-based methods include all the methods that represent a text with a graph and subsequently calculate properties of the graph, such as size or some measure of connectivity. The graph representations can be based on the \textit{co-occurrence} of words in the text, words being the nodes of the graph and neighbouring words being connected by a directed edge. Alternatively, the graphs may be based on a \textit{semantic} representation of the text, which can be obtained with semantic role labelling tools (SRL, \cite{gildea2002automatic}). The graphs can also be built at the level of sentences rather than words, as in Rhetorical Structure Theory (RST, \cite{mann1988rhetorical}) and the distribution of edge types may be used as a predictive feature.

% Reduced graph connectivity is typically regarded as indicative of positive FTD symptoms such as incoherence or derailment and flight of ideas, as if very little is said on any given topic it would result in a largely disconnected graph. On the other hand, very high connectivity may be indicative of negative FTD symptoms of repetitiveness or perseveration.

% Methods of obtaining graphs
Graph-based metrics represent the texts with a graph and then calculate properties of the resulting graph to use as predictive features. There are two main approaches to representing text with graphs being used in the field of psychosis detection. The first approach is based on co-occurrence, where the words are used as graph nodes and they are connected with an edge if they follow one another in a sentence. Many studies rely on software designed specifically for construction of such graphs, called SpeechGraphs \citep{mota2012speech, mota2014graph}. Alternatively, the graph may be formed on the basis of semantic connections, with words used as the nodes and semantic relationships between them as the edges \citep{nikzad2022does, nettekoven2023semantic}, with some studies showing that semantic graphs function better than co-occurrence graphs for some tasks \citep{nikzad2022does}.

% Graph metrics
The most popular graph characteristics of the graph used for psychosis detection include the numbers of nodes (N) and edges (E), as well as the number of nodes in the largest connected component (LCC)\footnote{A subgraph in which all nodes are connected by some path.} and largest strongly connected component (LSC)\footnote{A subgraph in which all nodes are connected by an edge.}. Some studies also measure the organization by comparing the size of LCC and LSC to that of random graphs of the same size and calculating z scores (LCCz and LSCz, respectively). Several studies report lower number of edges in the patient population \citep{mota2014graph, mota2016quantifying, mota2017thought, nikzad2022does}, as well as smaller largest strongly connected component and largest connected components \citep{mota2014graph, mota2016quantifying, mota2017thought, spencer2021lower, morgan2021natural, nikzad2022does}. \citet{nettekoven2023semantic} report lower numbers and smaller median size of connected components in patient populations. As for organization, the LSCz and LCCZ scores were reported to be lower, indicating more random-like graphs in patient populations \citep{mota2016quantifying, mota2017thought, spencer2021lower, morgan2021natural, nikzad2022does} with LSCz being more reliable in differentiating between the groups. Many report that lower values of these graph features are associated with negative PANSS \citep{mota2014graph, mota2016quantifying, nikzad2022does} and negative linguistic symptoms measured by TLI/TLC \citep{morgan2021natural, spencer2021lower, nikzad2022does}, though some report no relation with PANSS \citep{mota2012speech, argolo2023burnishing, nettekoven2023semantic}. Some studies also report lower average total degree \citep{mota2014graph} or average weighted degree \citep{nikzad2022does}\footnote{Other metrics include NP recurrence and distance between NPs \citep{palominos2023coreference}, graph centrality measures \citep{argolo2023burnishing}, number of nodes in the largest clique \citep{tang2022clinical, tang2023latent}, number of repeated and parallel edges \citep{mota2012speech, mota2014graph}, number of loops of lengths one, two, and three \citep{mota2012speech, mota2014graph}, and number of connected and strongly connected components \citep{argolo2023burnishing, nettekoven2023semantic}.}.

Some studies also use graph \textit{density}\footnote{2E/N(N - 1)}, which was reported by \citet{nikzad2022does} to be lower in patient populations, while \citet{mota2012speech}, \citet{mota2014graph}, and \citep{argolo2023burnishing} found no such differences. Graph \textit{diameter}\footnote{The length of the longest shortest path between the node pairs of a network.} was used in multiple studies but was reported to differ between groups only in one study \citep{mota2014graph}; and the same is true of \textit{average shortest path}\footnote{The average length of the shortest path connecting each pair of nodes in the graph.} \citep{mota2012speech, nikzad2022does, tang2022clinical, argolo2023burnishing}. 

Several papers also use various graph features as input for classifier models to predict negative symptoms \citep{mota2022happy, tang2023latent} and social cognition \citep{tang2022clinical}.

As many of these metrics, such as the total number of nodes and the number of nodes in the largest connected component, depend on verbosity, controlling for the effects verbosity is very important, and many studies adjust the metrics for the number of words or calculate graphs over windows of around 100 words.

The results for graph methods are summarized in table \ref{tab:method:graph} in appendix.

\textcolor{red}{I am not sure I am gonna use graph based metrics at all (it may be too much of a hassle)}

%----------------------------------------------------------------------------------------
%	SECTION 4
%----------------------------------------------------------------------------------------

\section{Language Model-Based Methods}
\label{sec:review:LM}

There is a great variety of language model-based methods of psychosis detection involving a variety of language models as well as many different ways of using these models. This part of literature review is organized according to the types of LM-based metrics and covers related choices. The section is structured as follows: subsection \ref{sec:review:LM:embeddings} covers the methods that use embeddings and subsection \ref{sec:review:LM:features} is dedicated to other applications of LMs. Subsection \ref{sec:review:LM:agrregation} is dedicated to the ways scores for separate units are aggregated into a single participant or text score. Subsection \ref{sec:review:LM:models} compares different LMs used. Finally, subsection \ref{sec:review:LM:interaction} discusses the ways in which metrics and models interact.

%-----------------------------------
%	SUBSECTION 1
%-----------------------------------
\subsection{Embedding-Based Methods}
\label{sec:review:LM:embeddings}
Most LM-based methods of psychosis detection operate over vectors (or embeddings) of words, phrases, or sentences. 

\subsubsection{Word-Based Methods}

While many apply word embedding based metrics to structured tasks, such as verbal fluency task or single-word associations to assess word similarity\footnote{\citep{elvevaag2007quantifying, holmlund2019updating, pietrowicz2019new, ku2021computational}}, this technique is less popular for less structured text with fully-formed sentences. Nevertheless, several studies report lower cosine similarity between consecutive words (\textit{word-based coherence}) in the patient population in several languages (Hebrew, \cite{bar2019semantic}; German and Chinese \cite{parola2022speech}) or observe a correlation of the scores with human coherence judgement \citep{xu2020centroid}, though some find no group differences \citep{liebenthal2022linguistic, argolo2023burnishing} or report the opposite direction of difference (Danish, \cite{parola2022speech}). Alternatively, one may calculate average similarity of words at a fixed distance (\textit{k-inter-word similarity}), as suggested by \citet{corcoran2018prediction}, who found lower k-inter-word similarities in CHR group as compared to controls at distances of 5 to 7, even after controlling for sentence length, though no correlation with symptom severity. \citet{parola2022speech} show similar results for Chinese and German, but higher k-inter-word similarity in Danish-speaking patient population as compared to control, and \citet{argolo2023burnishing} found no group differences on a CHR population. Some studies also assess similarity between all pairs of words (\textit{all-word similarity}, \citet{alonso2022language, alonso2022progressive, liebenthal2022linguistic}), but none find group differences in this metric and only one study finds a correlation with symptom severity \citep{alonso2022progressive}. Two studies also assess \textit{vector magnitude}, finding no differences between groups \citep{rezaii2019machine, liebenthal2022linguistic}. Similarity of each word vector to the \textit{centroid} of all word vectors, or \textit{cumulative centroid} of all words preceding a given word also were tested and proved moderately efficient in approximating human coherence judgements \citep{xu2020centroid, xu2022fully}. One paper also successfully used cosine similarity between word windows including or excluding different groups of connective words \citep{corona2022assessing}.

Another popular approach is the \textit{moving-window coherence} also known as \textit{sliding-window coherence}, where the metric is calculated as the average similarity between each word pair in a window of several words, rather than between sentences, and the window is moved by one word along the text. This method is particularly well adapted for shorter texts where the number of sentences might be insufficient to produce reliable coherence assessments \citep{panicheva2019semantic}. The studies typically assess various window sizes between 2 and 10, and the results vary greatly. While some report lower mean values in patient populations for windows of size 1 to 5 (on content words in Hebrew, \cite{bar2019semantic}) and 5 and 10 (for Chinese, \cite{parola2022speech}), others report higher mean moving-window coherence values for windows of size 5 and 10 \cite{alonso2022language, alonso2022progressive} or no difference at all (for window of size 8 on Dutch-speaking patient populations, \cite{dore2019quantification} and windows of size 5 and 10 on German and Danish-speaking patient populations \cite{parola2022speech}). Some also report higher variance of moving-window coherence values in Dutch-speaking patient populations \citep{voppel2021quantified, voppel2023semantic} with window sizes from 5 to 10, as well as lower maximum and higher minimum values for Russian speaking patient populations \citep{panicheva2019semantic} with window size of 4. 

The results for word-based LM methods are summarized in table \ref{tab:method:word-embedding} in appendix.

\subsubsection{Sentence or Phrase-Based Methods}

More commonly than on individual word vectors, papers focus on embeddings of sentences or phrases. 

% Coherence 

By far the most popular metric is so-called \textit{first-order coherence} or simply \textit{coherence}, which is calculated as the similarity between adjacent sentences or phrases. This metric was used in half the studies that relied on LM-based metrics with mixed results. It is important to note that while most studies report mean cosine similarity of all adjacent sentence pairs, some report minimal or maximal values, complicating the direct comparisons.

The first studies to adopt this metric actually used it as a feature in predictive models \citep{bedi2015automated, rosenstein2015language}, and this is frequently used today as well \citep{sarzynska2021detecting, tang2022clinical, tang2023latent}. While many report lower coherence scores in the patient population \citep{iter2018automatic, just2019coherence, morgan2021natural, ryazanskaya2020thesis}, some of these studies only observe this effect with few embedding methods among may tested settings \citep{iter2018automatic, just2019coherence} or report that the results are significantly affected by sentence length to the extent that the results are insignificant after controlling for it \citep{just2019coherence}. Some studies report finding no group difference at all, both for SDD \citet{iter2018automatic, just2020modeling} and CHR populations \citep{hitczenko2021understanding, bilgrami2022construct, haas2020linking}. The metric also reported to fail as a predictor of longitudinal outcomes \citep{just2023validation}.

As for correlation with symptoms, while some studies find correlation with PANSS, especially negative subscales \citep{ryazanskaya2020thesis, just2023validation}, others find no such relation but report correlation with SANS \citep{parola2022speech}. \citet{just2020modeling} report lower coherence in patients with positive FTD as opposed to patients without it, though no significant group difference with controls was found. On CHR populations, no correlation with symptom severity was observed \citep{hitczenko2021understanding}.

Several validation studies report that the coherence metric correlates manual assessments of coherence \citep{xu2020centroid, xu2022fully} and maximal coherence scores were reported to correlate with TLC subscales of tangentiality, derailment, circumstantiality, loss of goal, poverty of speech, pressure of speech \citep{bilgrami2022construct}.

Altogether, despite the popularity of this metric there is still no consensus on whether it can serve as a reliable psychosis predictor, as the results vary across models \citep{iter2018automatic, just2019coherence, ryazanskaya2020thesis} and languages \citep{just2020modeling, parola2022speech}, and the metric seems to fail for CHR populations \citep{hitczenko2021understanding, bilgrami2022construct, haas2020linking}.


Additionally, \citet{bedi2015automated} proposed \textit{second-order coherence}, which compared sentences one sentence apart rather than consecutive sentences. The authors used this metric in a classifier model with very high accuracy \citep{bedi2015automated}, and a similar approach was since applied to Polish \citep{sarzynska2021detecting}. \citet{parola2022speech} also tested second order coherence in Chinese, Danish, and German, and found it to be significantly lower in the patient populations. \citet{morgan2021natural} also try to assess \textit{repetitiveness} by calculating maximal similarity among all sentence pairs but find no significant difference in this metric between the groups.


% Global coherence

Another family of approaches tries to approximate how well each sentence is related to the main topic, which can be seen as assessing \textit{global} rather than \textit{local} coherence. One approach is to assess how close the vector of an entire response is to the vectors of responses produced by other participants (\textit{group global coherence}) or to a gold standard response (\textit{gold standard global coherence}). These two approaches is especially useful for picture-elicited texts or story retellings, as the contents could reasonable be expected to be highly similar for most responses. The \textit{group global coherence} was introduced in the pioneering work by \cite{elvevaag2007quantifying}, where the similarity to other patient responses was successfully used to predict human ratings of organizational structure, tangentiality, and content, and it was replicated in \citet{elvevaag2010automated}. The method was also applied to Russian material, correlating with local coherence judgements \citep{ryazanskaya2020automated}, differentiating between the groups, and correlating with PANSS scores \citep{ryazanskaya2020thesis}. The similarity to a gold standard description correlated with judgements of local coherence and violations of completeness on Russian material \citep{ryazanskaya2020automated} and was reported to be able to differentiate between the groups on English-speaking patient populations \citep{morgan2021natural}\footnote{\citet{nettekoven2023semantic} utilize this metric but do not report on its efficiency.}. An alternative approach to global coherence was proposed by \citet{xu2020centroid}, where each sentence is compared to the centroid (average) of all sentence vectors (\textit{centroid global coherence}) or to the centroid of all sentences preceding a sentence (\textit{cumulative centroid global coherence}). Both metrics were shown to correlate with human coherence judgements \citep{xu2020centroid, xu2022fully} as well as to predict TALD scores \citep{xu2022fully}. The metrics were also successfully applied to Russian, being able to differentiate between the groups and correlating with PANSS scores \citep{ryazanskaya2020thesis}; as well to German, where the centroid global coherence was shown to correlate with negative, disorganized, and exited PANSS subscales \citep{just2023validation}, though with no ability to predict longitudinal outcomes.

% Tangentiality

Finally, several metrics were developed to assess \textit{tangentiality}, meaning the relevance to the topic under discussion or the question asked. The most popular approach to tangentiality assessment was introduced in the pioneering work by \cite{elvevaag2007quantifying}. The researchers used the sliding window and calculated the similarity of each window to the question posed by the interviewer. Then, they calculated the slope, to assess how quickly each participant moved away from the topic. They found a significant correlation between the slope of the similarity values and the blind human ratings of tangentiality, as well as greater variance of the values at higher window sizes and in patients with high FTD.  While some studies report higher values of tangentiality (steeper slopes) in the patient populations \citep{iter2018automatic, tang2021natural}, others find no significant group differences both in English \citep{hitczenko2021understanding, morgan2021natural} and in other languages (German, \cite{koranova2017analyzing, just2019coherence} and Dutch, \cite{dore2019quantification}). The same studies also report no correlation with symptom severity both in SDD \citep{dore2019quantification} and on CHR populations \citep{hitczenko2021understanding, morgan2021natural}. \citet{koranova2017analyzing} and \citet{just2019coherence} also assess mean similarity to the question rather than the slope, yet also find no group differences.

Table \ref{tab:def:LM} provides formulae for the metrics described above. The results for sentence-based LM methods are summarized in table \ref{tab:method:sent-embedding} in appendix.

\begin{table}[]
\resizebox{\textwidth}{!}{%
\begin{tabular}{lll}
\hline
\textbf{Metric Name} & \textbf{Definition} & \textbf{Introduced} \\ \hline
(First-Order) Coherence                    & $agg_{i=1}^{N}(cossim(S_i, S_{i+1}))$ & \cite{bedi2015automated} \\
Second-Order Coherence                   & $agg_{i=1}^{N-1}(cossim(S_i, S_{i+2}))$ & \cite{bedi2015automated} \\
Repetitiveness                           & $max_{i,j=1, i \neq j}^{N}(cossim(S_i, S_j))$ & \cite{morgan2021natural} \\
Moving Window Coherence                  & \begin{tabular}[c]{@{}l@{}}$agg_{i=1}^{M-k}(mean(cossim(W_x, W_y)))$ \\ and $ 0 < y-x \leq k;  i \leq x, y < M $ \end{tabular} & \begin{tabular}[c]{@{}l@{}}\cite{bar2019semantic} \\ \citet{panicheva2019semantic} \end{tabular} \\
K-Inter Word Similarity                    & $agg_{i=1}^{M-k}(cossim(W_i, W_{i+k}))$ & \cite{corcoran2018prediction} \\
Group Global Coherence                   & $cossim(R_a, mean_{b=1}^{Q}(R_b))$ & \cite{elvevaag2007quantifying} \\
Gold Standard Global Coherence           & $cossim(R_a, R_{gold})$ & \citet{ryazanskaya2020automated} \\
Centroid Global Coherence                & $agg_{i=1}^{N}(cossim(S_i, mean_{j=1}^{N}(S_j)))$ & \cite{xu2020centroid} \\
\begin{tabular}[c]{@{}l@{}}Cumulative Centroid \\  Global Coherence \end{tabular}     & $agg_{i=1}^{N}(cossim(S_i, mean_{j=1}^{i}(S_j)))$ & \cite{xu2020centroid} \\
Slope Tangentiality                      & $slope_{i=1}^{N}(cossim(S_q, S_i))$ & \cite{elvevaag2007quantifying}\footnote{Some use moving windows rather than sentences, with $slope_{i=1}^{M-k}(mean(cossim(W_x, W_y)))$ where $x-y \leq k;  i \leq x, y < M $.} \\
Q-similarity Tangentiality               & $agg_{i=1}^{N}(s_q, s_i)$ & \cite{koranova2017analyzing} \\ \hline
\end{tabular}}
\captionsetup{width=\textwidth}
\caption[Definitions of the embedding-based methods]{\label{tab:def:LM} Definitions of the embedding-based methods. \\
$S$ - sentence embedding; $W$ - word embedding; $R$ - response embedding; $cossim$ - cosine similarity; $agg$ - aggregation scheme (such as mean, max, min, or median); $slope$ - slope of a linear regression function; $S_q$ - question embedding; $R_{gold}$ - gold standard description embedding; $M$ - number of words; $N$ - number of sentences; $Q$ - number of people in the group. Score aggregation is discussed in more detail below (\ref{sec:review:LM:agrregation}).}
\end{table}

Among LM-based methods, I intend to test \textcolor{red}{coherence} as the most frequently used metric, as well as \textcolor{red}{second order coherence and centroid global coherence}, as they are some of the more consistently well-functioning metrics. \textcolor{red}{I may also test all word and k-inter-word coherence, but I may not.} Unfortunately, the data at my disposal does not allow for testing tangentiality, as it is only well adapted for semi-structured interviews with longer responses.

%-----------------------------------
%	SUBSECTION 2
%-----------------------------------
\subsection{Feature-Based Methods}
\label{sec:review:LM:features}
An alternative group of methods does not rely on cosine similarity of embedded text, but rather use the language models to generate predictive features or fine-tune the models for the psychosis prediction tasks.

An early approach was to use an embedding of the entire text as a feature in a classifier model with no fine-tuning of the language model \citep{elvevaag2010automated, rosenstein2015language}, yielding relatively high classification accuracy of 0,72 - 0,83. Recently, BERT embeddings were successfully used to distinguish between SDD patients and healthy controls \citep{srivastava2022p473}. Now, the studies that have enough data for such a procedure, \textit{fine-tune transformer-based language models} \citep{wouts2021belabbert, aich2022towards, shriki2022masking}. While some report high accuracy with fine-tuned BERT (0,84 in Hebrew, \cite{shriki2022masking}) or fine-tuned RoBERTa (0,76 in Dutch, \cite{wouts2021belabbert}), others report that fine-tuning of various transformer models performs very poorly compared to feature-based methods, reporting 0,6 accuracy for MentalRoBERTa; 0,38 for MentalBERT, and 0,33 for BERT base as compared to 0,70-0,96 achieved by a random forest classifier over simpler features on an English-speaking population \citep{aich2022towards}. One of the challenges to the fine-tuning approach is the very small sample size of most studies (with mode sample size of 20 patients and 21 controls). \citet{aich2022towards} have the largest sample size, with 247 patients and 110 controls, and yet they report low accuracy for fine-tuned models. Some combat the problem of sample size by using small chunks of each text as fine-tuning material \citep{wouts2021belabbert}, yet that still requires a significant sample size for sufficient fine-tuning, which is unattainable in many clinical settings, and the models trained on mined mental health related texts, such as MentalBERT, seem to fail to transfer to the speech domain \citep{aich2022towards}.

Another type of metric that has been proposed by \citet{mitchell2015quantifying} is \textit{perplexity} or \textit{surprisal}. This metric is a model assessment of how likely a given text fragment is. Perplexity is typically used to assess the quality of the model, as the model should give low perplexity or surprisal scores on real texts. In other words, the model should not be very surprised by normal real world texts. Some hypothesized that as psychotic speech is atypical, it could receive higher perplexity or surprisal scores, while others suggested that stereotyped speech would result in lower perplexity values. While \citet{mitchell2015quantifying} apply perplexity to psychosis detection, they find no group differences in perplexity. In contrast, \citet{srivastava2022increased} report increased perplexity in SZ as compared to CHR and controls, and both \citet{vail2018toward} and \citet{girard2022computational} find an association between increased perplexity and higher PANSS scores in psychotic disorder populations\footnote{\citet{colla2022semantic} focus on the method itself and run an in-depth analysis of perplexity metric variants applied to Alzheimer Disease detection.}. On the other hand, \citet{jeong2023exploring} find no association of BERT surprisal scores with PANSS, though they report that it is positively correlated with pressure of speech, circumstantiality, illogicality, tangentiality and negatively with poverty of speech as measured by TLC, SANS, and SAPS.

Finally, some researchers suggested that one of the BERT training tasks, namely, \textit{next sentence prediction} could be utilized for psychosis detection. Like perplexity, next sentence prediction assesses the likelihood of the text, but rather than likelihood of one sentence, it measures how likely the two sentences are to follow one another. The model is trained to differentiate between sentences that did occur sequentially from the sentence pairs that were sampled randomly. For highly unpredictable speech such as is sometimes seen in psychosis, one could expect lower values of next sentence prediction layer of a BERT model. The metric was first used for psychosis detection by \citet{hitczenko2021understanding} with no significant differences found between healthy controls and CHR population. Subsequently, it was applied to SZ patients, also yielding no group difference \citep{tang2021natural} and no correlation with PANSS, though sentence likelihood was negatively associated with derailment, illogicality, and circumstantiality as measured by TLC, SANS, and SAPS \citep{jeong2023exploring}.

The results for feature-based LM methods are summarized in table \ref{tab:method:feature} in appendix. \textcolor{red}{Among feature-based metrics, I could assess perplexity and next-sentence prediction, but I am not sure, as the results are quite mixed. Fine-tuning is out of the question, as it requires more data than I have at my disposal.}

%-----------------------------------
%	SUBSECTION 3
%-----------------------------------
\subsection{Score Aggregation}
\label{sec:review:LM:agrregation}
As many of the embedding based methods produce number of scores, one for each sentence, for a pair of sentences, or for each window, multiple ways of aggregating these scores into a single score for the entire text can be utilized. While three quarters of the examined studies used mean of the scores, many also tested other approaches, such as minimum\footnote{\cite{bedi2015automated, iter2018automatic, corcoran2018prediction, panicheva2019semantic, haas2020linking, ryazanskaya2020thesis, xu2020centroid, morgan2021natural, sarzynska2021detecting, voppel2021quantified, bilgrami2022construct, corona2022assessing, xu2022fully, voppel2023semantic}} or maximum\footnote{\cite{corcoran2018prediction, panicheva2019semantic, haas2020linking, ryazanskaya2020thesis, morgan2021natural, bilgrami2022construct, corona2022assessing, voppel2023semantic}} of the scores. 

Minimum values in such metrics as first-order coherence are used frequently, as they were reported to perform well for group differentiation in an early work in the field \citep{bedi2015automated}. The results, however, are mixed with the minimum being utilized for a variety of metrics, and some reporting higher minimum values in patient populations \citep{panicheva2019semantic, corona2022assessing} and negative association with clinical tangentiality assessment \citep{bilgrami2022construct}, and others reporting lower minimum values in patient populations \citep{bedi2015automated, corcoran2018prediction, iter2018automatic, ryazanskaya2020automated}, and still others finding no group differences on CHR populations \citep{haas2020linking, bilgrami2022construct}. The minimum scores were also reported as highly useful classifier features in some studies \citep{bedi2015automated, xu2020centroid, xu2022fully, sarzynska2021detecting}.

While some report no difference in \textit{maximum} of scores \citep{haas2020linking, morgan2021natural, voppel2021quantified, bilgrami2022construct}, others report lower maximum cosine-similarity-derived scores in patient populations \citep{corcoran2018prediction, panicheva2019semantic, ryazanskaya2020thesis, corona2022assessing} and positive association with clinical tangentiality assessment \citep{bilgrami2022construct}.

Some works also utilize standard deviation or variance of the scores as a predictive feature\footnote{\cite{bedi2015automated, corcoran2018prediction, panicheva2019semantic, haas2020linking, hitczenko2021understanding, sarzynska2021detecting, voppel2021quantified, corona2022assessing, voppel2023semantic}}. While some report no difference in this aggregate for CHR populations \citep{haas2020linking, hitczenko2021understanding}, others find higher variance values \citep{voppel2021quantified, voppel2023semantic} and still others find lower values of variance in patient populations \citep{corcoran2018prediction}.

Other aggregation methods include median\footnote{\cite{bedi2015automated, sarzynska2021detecting, corona2022assessing, parola2022speech}}, 10 and 90 percentile values \citep{corcoran2018prediction, panicheva2019semantic}, range \citep{corona2022assessing}, interquartile range \citep{parola2022speech}, and sum \citep{jeong2023exploring}.

The use of different averaging approaches is listed in table \ref{tab:method:score-averaging} in appendix.

A simulation study, conducted by \citet{fradkin2023theory}, found that `whereas previous studies attempted to capture more complex dynamics of speech disorganization by accounting for how semantic distances vary within a narrative, our simulations showed that these alternative metrics are, in most cases, less sensitive than simple averaging'. \citet{parola2022speech} also report that among many tested options, `median and interquartile range more robustly present independent measures of mode and variance of the distribution, respectively'.

Overall, there is most evidence for mean and minimum values serving as a successful aggregation technique for finding group differences. \textcolor{red}{To limit the number of comparisons, only the mean will be used in the present work.}

% xu 2020 
% Each metric produces a series of similarity calculations, one for each comparison it makes. For example, the sequential word-level metric produces a cosine value for each pair of neighboring words, and the centroid-based metrics produce a cosine value for the comparison between each independent unit and the centroid. Thus, the output for every metric was an array of cosine values. We then calculated the minimum and mean value of the array to evaluate their utility as transcript-level coherence scores. Our motivation for doing so was that previous studies suggested the minimum and mean of the cosine array were effective in representing coherence \citet{bedi2015}
% Aggregation: The mean and minimum aggregation methods were evaluated by comparing the number of coherence metrics that performed best in terms of ROC curve AUC and Spearman correlation with each method. With ROC curve AUC, nine out of thirteen coherence metrics performed better when summarized by the minimum method. The 4 exceptions were the centroid metric at phrase level and the cumulative centroid metric at word, phrase, and weighted sentence levels. With Spearman correlation, the mean performed better than the minimum for only two of thirteen coherence metrics: centroid and cumulative centroid both at phrase level. Because of the generally better performance of the minimum, we report results with this approach to aggregation for the remainder of the paper.

% xu 2022
% and compare this with the typical approach of aggregating coherence estimates across transcripts: taking the minimum coherence score. This approach predominates in prior work on automated estimates of coherence in the context of thought disorganization [5 Elvevåg 2007], [6 Elvevåg 2010], [7 bedi 2015], [8 Corcoran 2018], underlies key validation studies in this area [5  Elvevåg 2007], [7 bedi 2015], and has been shown to outperform a range of other aggregate statistics in its agreement with human judgment and utility as a predictive feature of the onset of psychosis in high-risk individuals [7 bedi 2015], [18 Xu 2020].

% fradkin 2023
% Then, to obtain a single derailment metric for each narrative, we followed a convention used in previous studies by calculating either the mean, minimum, maximum, or variance of these distance measures (Figure 1).
% Finally, whereas for the above results we calculated narrative-level derailment by averaging semantic distances across all word pairs or sentence pairs, previous studies have used a variety of alternative aggregation methods, focusing on variability or extreme semantic distances (Figure 1). Critically, as shown in Figure 6, the benefit of using such alternative methods has been small and inconsistent. These results suggest that researchers may prefer to focus on the average (NLP-measured) derailment of narratives or otherwise choose an aggregation method based on the hypothesized mechanism and measure of interest.
% Finally, whereas previous studies attempted to capture more complex dynamics of speech disorganization by accounting for how semantic distances vary within a narrative, our simulations showed that these alternative metrics are, in most cases, less sensitive than simple averaging. 

% parola et al 2022
% We opted to use median of coherence measures, even if previous studies used a wide variety of descriptors (e.g. standard deviation, maximum, minimum, etc.) because it is more robust to measurement errors (see SM4 for details). To better identify the impact of this choice, we also report in SM5 additional analyses for interquartile range (IQR) and minimum of the various coherence measures.
%  SM5: We opted to use median and interquartile range of semantic measures, contrary to more commonly used mean, standard deviation and range, because they are more robust to outliers and provide more complementary information in case of long tails in the distribution described. Median and interquartile range more robustly present independent measures of mode and variance of the distribution, respectively.


%-----------------------------------
%	SUBSECTION 4
%-----------------------------------
\subsection{Language Models}
\label{sec:review:LM:models}
It is important to note that different studies use different language models and different ways of acquiring sentence embeddings from them which might have a significant influence on the study results. 

\subsubsection{Non-Contextualized Embeddings}

The pioneering works in the area of psychosis detection, such as \citet{elvevaag2007quantifying, elvevaag2010automated}, used latent semantic analysis models (\textit{LSA}) which work by applying singular value decomposition over word co-occurrence counts in a large collection of documents \citep{landauer1998introduction}. The resulting vectors could be used to represent the distributional semantics of the words, as the words that occurred in similar types of context are represented with similar vectors. The semantic similarity is often measured with cosine distance between the vectors, the metric known as \textit{cosine similarity}. The downside of LSA is that singular value decomposition is a very computationally expensive operation especially for large corpora. Additionally, the resulting vector representations are not context-aware and cannot distinguish between different meanings of the same word or between homonyms. Nevertheless, LSA has been used in many psychosis detection works, either as the only model \citep{elvevaag2007quantifying, elvevaag2010automated, rosenstein2015language, bedi2015automated, haas2020linking} or in comparison with other models \citep{iter2018automatic, xu2020centroid, hitczenko2021understanding, tang2021natural, tang2023latent}. While the early works report that LSA was useful as a feature in a classifier \citep{elvevaag2007quantifying, elvevaag2010automated, rosenstein2015language, bedi2015automated}, the model has failed to differentiate healthy controls from CHR population \citep{hitczenko2021understanding, haas2020linking}. Additionally, while LSA has been reported to outperform other non-contextual embeddings (word2vec) when trained on a small corpus \citep{xu2020centroid}, given sufficient training data, LSA was outperformed by word2vec \citep{iter2018automatic, xu2020centroid}.

\textit{Word2vec} is by far the most popular embedding model in the field of psychosis detection. Introduced by \citet{mikolov2013distributed}, word2vec relies on neural network architecture to create a representation of the words that can predict if the words are likely occur in the same places in a text. The resulting vector representations of the words can be used to assess semantics of text using cosine similarity, though the representations are also not context-aware. With modern computational capacities and large corpora, word2vec is quite easy to train, and ready-to-use word2vec representations trained on internet common crawl as well as Wikipedia are \href{https://fasttext.cc/docs/en/crawl-vectors.html}{available}\footnote{https://fasttext.cc/docs/en/crawl-vectors.html} for 157 languages \citep{bojanowski2017enriching}. This makes word2vec a very attractive model choice for many researchers, and it was, in fact, used in half of the articles that relied on LM-metrics, and for more than a half of the articles that used word2vec it was the only model they used. 

% PAROLA et al 2022 SM Characteristics and cross-linguistic comparability of the fastText models 
% The fastText models rely on the works of Bojanowski et al. (2017) and Grave et al. (2018). The fastText models (Bojanowski et al., (2017) are trained on one large source of text data, which is the free online encyclopedia Wikipedia. The size of the articles available on Wikipedia differs across the different languages, for an overview see: https://meta.wikimedia.org/wiki/List_of_Wikipedias. Grave et al. (2018) tested the performance of the pre-trained fastText word vectors across 10 languages, including Chinese, German, and Finnish (similar in number of Wikipedia articles to Danish). The performance was extremely good across all languages but Hindi which had the smallest amount of Wikipedia data available (53% of the Danish, our smallest language).  The choice to rely on Wikipedia for the model training is motivated by the fact that Wikipedia is one of the largest available text sources, which is curated, and roughly comparable across a large number of languages. This makes it an ideal source of training material for cross-linguistically scalable methods. However, alternative methods, such as topic modeling (Hwang et al., 2020; Maupomé, & Meurs 2018), semantic clustering (e.g., Rezaii et al., 2019), and multilingual transformers (e.g., Wouts et al., 2021; Tang et al., 2021) fine-tuned on smaller ad hoc corpora, might be more sensitive to differences in coherence measures between patients and controls.

Quite a few papers compared word2vec to newer models, such as GloVe \footnote{\citep{iter2018automatic, just2019coherence, tang2022clinical, tang2023latent, just2023validation}}, sent2vec \citep{iter2018automatic, just2019coherence, hitczenko2021understanding}, ELMo \citep{ryazanskaya2020thesis, hitczenko2021understanding, sarzynska2021detecting}, and BERT \citep{ryazanskaya2020thesis, hitczenko2021understanding, xu2022fully}. While some report that the choice of the model does not affect the outcomes \citep{hitczenko2021understanding, fradkin2023theory}, other find significant differences between the models, interaction between models, metrics, and tasks. While \citet{iter2018automatic, just2023validation} report word2vec outperforming GloVe, \citet{just2019coherence} observe the opposite. ELMo and BERT were reported to outperform word2vec in symptom severity assessment \citep{ryazanskaya2020thesis} but proved equally unsuccessful as word2vec in distinguishing CHR from controls \citep{hitczenko2021understanding}. Additionally, BERT and its variants were shown to outperform word2vec in predicting human coherence judgement \citep{xu2022fully}. Finally, while \citet{iter2018automatic} report that word2vec outperforms \textit{sent2vec}\footnote{\textit{sent2vec}\citep{moghadasi2020sent2vec} was only used in a few papers in the field in comparison with other methods and is less popular than transformer-based architectures.}, both \citet{just2019coherence} and \citep{hitczenko2021understanding} show that the two are equally inefficient for psychosis detection.

% LSA -  Iter et. al 2018, Xu et al. 2020, Hitczenko et al. 2021, Tang et al. 2022, Tang et al. 2023
% GloVe - Iter et. al 2018, Just et al. 2019, Tang et al. 2022, Just et al. 2023, Tang et al. 2023
% ELMo - Ryazanskaya 2020, Hitczenko et al. 2021, Sarzynska-Wawer et al. 2021
% BERT+ - Ryazanskaya 2020, Hitczenko et al. 2021, Xu et al. 2022
% sent2vec - Iter et. al 2018, Just et al. 2019, Hitczenko et al. 2021

\textit{GloVe} is conceptually similar to word2vec and is another non-contextualized word embedding model \citep{pennington2014glove}. It is incorporated into such tools as \href{https://spacy.io/}{SpaCy}\footnote{https://spacy.io/} and \href{https://www.covingtoninnovations.com/software/CoVec-manual.pdf}{CoVec}\footnote{https://www.covingtoninnovations.com/software/CoVec-manual.pdf}. The papers that use exclusively GloVe embeddings report quite limited success, finding no group differences \citep{just2020modeling, alonso2022language, alonso2022progressive}, tough \citet{alonso2022progressive} report correlation with PANSS and \citet{just2020modeling} report lower coherence values in high FTD patients.

For perplexity assessments, several papers used \textit{trigram	models} which directly assess the co-occurrence likelihood of three word sequences, rather than modeling words with vectors \citep{mitchell2015quantifying, vail2018toward, girard2022computational}. This method was reported to correlate with symptom severity in one study \citep{vail2018toward}, while others found no significant effect \citep{girard2022computational} or group difference in perplexity \citep{mitchell2015quantifying}.


\subsubsection{Sentence Embedding Aggeragation}

Another significant factor in word embedding models is the way the vectors representing individual words are combined to obtain a single sentence or phrase representation. By far the most popular technique is \textit{averaging} all the word vectors in a sentence or window, yet it has been reported by several independent studies times to be noisy and confounded \citep{fradkin2023theory} as well as to produce undesirable connection between cosine similarity metrics and sentence length \citep{hitczenko2021understanding, parola2022speech, fradkin2023theory}. The reason for this is that averaging static word vectors with no weights results in a more generic and meaningless representation for longer sentences and it fails to reflect that function words, such as articles, might contribute little to the meaning of a sentence while driving all sentence vectors closer to a meaningless average. Therefore, many researchers utilize such weighing schemes as \textit{TF-IDF}, where words are weighted inversely proportional to their corpus frequency and directly proportional to the word frequency in the sentence. This method is used in quite a few papers, with several reporting successful application over LSA \citep{iter2018automatic} or word2vec \citep{just2019coherence, xu2022fully}, others using it in classifier models \citep{ryazanskaya2020automated, tang2023latent}, and some reporting no group difference \citep{just2020modeling, hitczenko2021understanding}. \citet{xu2020centroid} actually find no weighting more efficient that TF-IDF in predicting human coherence judgements, but unlike other studies that use weighted average,  \citet{xu2020centroid} use weighted sum of the vectors. Another method used in several studies is smooth inverse frequency (\textit{SIF}), introduced by \citet{arora2017simple}. It combines frequency weighting with removing the common meaningless component, producing better-performing sentence representations. Metrics over this representation were shown to successfully differentiate between groups by \citet{iter2018automatic, ryazanskaya2020thesis, morgan2021natural, nettekoven2023semantic}, but could not help to differentiate healthy controls from CHR \citep{hitczenko2021understanding} and were not correlated with symptom severity \citep{iter2018automatic, ryazanskaya2020thesis, hitczenko2021understanding, morgan2021natural}. Unlike \citet{iter2018automatic}, who find both TF-IDF and SIF somewhat efficient, \citet{just2019coherence} only finds group differences with TF-IDF, but not SIF or mean averaging. All in all, one could cautiously advise against the use of unweighted average, as they are more prone to the problems of meaningless similarity and length dependence. The use of different word embedding sentence averaging methods is listed in table \ref{tab:method:word-embedding-averaging} in appendix.

\subsubsection{Contextualized Embeddings}

Contextualized embeddings such as \textit{ELMo} and \textit{BERT} are capable of representing the words in context and differentiating between different senses of words dependent on the surrounding context. Additionally, these embeddings are capable of representing the entire sentences as a whole rather than individual words. ELMo uses bi-directional LSTM units to obtain contextualized word and sentence embeddings \citep{peters2017semi}. This model was used in several studies with some success, as \citet{ryazanskaya2020thesis} report comparable performance of ELMo and BERT in detecting group differences and correlation with PANSS scores, \citet{sarzynska2021detecting} use ELMo in a classifier model with 0,8 accuracy, and \citet{srivastava2022increased} use biLSTM-based perplexity as a feature for predicting symptom severity. On the other hand, \citet{hitczenko2021understanding} report ELMo equally unable as all other models to differentiate between CHR and healthy controls.

\textit{BERT}, introduced by \citet{devlin2018bert}, along with other transformer architectures, such as RoBERTa, is commonly used in more recent works dedicated to psychosis detection. Many different approaches to using BERT have been utilized over the years, as BERT can produce different features, including second-to-last layer of hidden state output and CLS token, that can be used for representing the entire sentence, or one might use word embeddings and encode the sentence as sum of the token embeddings that form the sentence \citep{xu2022fully}. The results of using BERT embedding are mixed, some reporting successfully differentiating between the groups or scores correlating with symptoms \citep{ryazanskaya2020automated, xu2022fully, srivastava2022p473} and similarly good results observed for SentenceBERT model \citep{xu2022fully}, introduced by \citet{reimers2019sentence}. Others report lack of group difference on CHR populations \citep{hitczenko2021understanding, bilgrami2022construct} or no correlation of symptom severity with next sentence prediction and surprisal metrics \citep{jeong2023exploring}. As reported above, fine-tuning BERT models, including base BERT, RoBERTa \citep{liu2019roberta}, MentalBERT, and Mental-RoBERTa \citep{ji2021mentalbert} also shows mixed results with some reporting high \citep{wouts2021belabbert, shriki2022masking} and some low classification accuracy \citep{aich2022towards}. The same mixed findings also hold for other methods that rely on BERT models, namely, to the next-sentence prediction and perplexity methods discussed above, some reporting successful application while others find no significant group differences or symptom correlations. It is important to note that such large models might be more sensitive to the computational restrictions than smaller models \citep{kaplan2020scaling}, and different BERT models were reported to show different performance levels in psychosis detection \citep{aich2022towards}. The upside of complex models is that it is possible to incorporate other types of information, such as audio or temporal information into the model \citep{xu2022fully, wouts2021belabbert}.

% BERT (second-to-last layer of hidden state output, CLS token, sum of the token embeddings that form the sentence) 
% SentenceBERT Xu 2022
% BERT base Xu 2022, ryazanskaya 2020, Hitczenko et al. 2021, Srivastava et al. 2022, Bilgrami et al. 2022 
% fine-tuned BERT Shriki et al., 2022, Aich et al. 2022, Wouts et al. 2021
% fine-tuned RoBERTa Wouts et al. 2021
% MentalBERT \& and Mental-RoBERTa Aich et al. 2022

% next sent pred The metric was first used for psychosis detection by Hitczenko, Mittal, and Goldrick (2021) with no significant differences between healthy controls and CHR population. Subsequently, it was applied to SZ patients, also yielding no group difference (Tang et al., 2021) and no correlation with PANSS, though sentence likelihood was negatively associated with derailment, illogicality, and circumstantiality as measured by TLC, SANS, and SAPS (Jeong et al., 2023).

% While some report high accuracy with fine-tuned BERT (0,84 in Hebrew, \cite{shriki2022masking}) or fine-tuned RoBERTa (0,76 in Dutch, \cite{wouts2021belabbert}), others report that fine-tuning of various transformer models performs very poorly compared to feature-based methods, reporting 0,6 accuracy for MentalRoBERTa; 0,38 for MentalBERT, and 0,33 for BERT base as compared to 0,70-0,96 achieved by a random forest classifier on English-speaking population \citep{aich2022towards}. 


The use of different language models is listed in table \ref{tab:method:LM-models} in appendix. \textcolor{red}{To limit the number of comparisons, I intend to use word2vec with TF-IDF sentence weighting and BERT sentence embeddings as two of the most frequently used and structurally different models. These will be taken two represent non-contextualized and contextualized embeddings.}

%-----------------------------------
%	SUBSECTION 5
%-----------------------------------

\subsection{Model-Metric Interactions}
\label{sec:review:LM:interaction}

Taking into account the wide choice of both models and metrics that use them, it is important to add that most studies that compare both several models and several metrics that utilize them, find significant model-metric interaction. \citet{iter2018automatic} report that the coherence metric was different between the groups only with word2vec embeddings and SIF averaging, while for tangentiality both word2vec with SIF and LSA with TF-IDF produced classifying scores, and report GloVe and sent2vec not working for any metric. Both \citet{just2019coherence} and \citet{hitczenko2021understanding}, on the other hand, observe that only coherence calculated with GloVe model and TF-IDF weighting could distinguish the groups, and even this was insignificant after correcting for the effect of length, with all other LM-metrics and models showing no effect at all. \citet{just2023validation} report that both local and global coherence calculated with GloVe or word2vec with mean averaging could predict negative symptom severity, and only word2vec model correlated other subscales. \citet{ryazanskaya2020thesis} reported that on one of the elicitation tasks metrics could worked better with word2vec with SIF or ELMo while on another task BERT showed better performance in distinguishing the groups, and the results were highly inconsistent across tasks, metrics, models, and averaging. Both \citet{xu2020centroid} and \citet{xu2022fully} also report some metrics performing better with one model than another without clear indication of one model being universally better. While \citet{xu2022fully} report SentenceBERT cumulative centroid global coherence as the best metric for both TALD and correlation with symptom severity, they also show that best metrics in approximating human coherence judgement depend on model selection BERT CLS token and Sentence BERT working for cumulative centroid global coherence, and BERT word vector summation along with Sentence BERT for centroid global coherence. Nevertheless, it could cautiously be suggested that contextualized embeddings, such as ELMO and BERT, should be favoured, as they become more and more available for a wider range of languages, as they could be less prone to length dependence stemming from word averaging \citep{fradkin2023theory}.

\citet{just2023validation} suggest that `the choice of NLP model should not be arbitrary', showing that even models trained on the same material can still yield different results depending on the model architecture alone. And stating that `the effect of different embedding models both intra- and cross-linguistically requires further investigation'.

% Just et al 2023. 
% While these results should be interpreted with caution, it might imply that the word2vec models outperformed the GloVe models in calculating coherence scores that were associated with clinically relevant outcomes in this study. This corresponds to findings by Iter et al. (27) who found significant group differences between patients and controls for the word2vec incoherence model, not GloVe, but contrasts our own previous study that found a superiority of the GloVe model in prediction of psychopathology in NAP (16 just et al 2019). These results indicate that the choice of NLP model should not be arbitrary. It has to be taken into account that different models, that is models with different architecture (e.g., GloVe vs. word2vec) as well as models trained on different corpora, produce different word vectors – this could explain the different results between this current and the previous study (16), having used different training data as well as different preprocessing and a more sophisticated sentence annotation. A concern would be that the chosen model has a stronger effect on the coherence scores than the difference between groups or coherence metric used (e.g., local vs. global). In this study, all coherence scores were still highly correlated with each other (see Table 2). The reason for this is, probably, the fact that the two models are both trained on the same material, that is German Wikipedia. If models are trained on different material, correlation between them can be low. This might be one of the key challenges of cross-linguistic application of NLP methods (24 Parola 2022) and the reason for limited replicability of the results within one language across models (25 Hitczenko 2021) and studies (16 just et al 2019, 25 hitzenko 2021, 26 Panicheva litvinova 2019). Nevertheless, our results show that models trained on the same material can still yield different results. The effect of different embedding models both intra- and cross-linguistically therefore requires further investigation.



%----------------------------------------------------------------------------------------
%	SECTION 5
%----------------------------------------------------------------------------------------

\section{Cross-Group Comparison of Methods}
\label{sec:review:cross-group}

Having discussed various methods in detail, I will now compare the groups of methods by aggregating the results of studies that include several methods of different types. It is important to note, that the results between these studies might not be directly comparable, as different studies may be using different metrics from the same group. Therefore, even if the studies report opposite trends it does not necessarily imply contradictory results.

The most widely used groups are language model-based methods and syntactic methods. Table \ref{tab:comparison:LM-Synt} summarizes the results reported by studies comparing some LM-based to some syntactic methods. Some studies find both groups efficient in differentiating between the groups \citep{bar2019semantic} or correlating with clinical scales \citep{argolo2023burnishing}, and others find that neither group of methods can efficiently distinguish CHR from controls \citep{haas2020linking} or assess CHR symptom severity \citep{bedi2015automated, corcoran2018prediction}. Some also find mixed results with only some metrics from both groups functioning well in identifying group differences \citep{tang2021natural} or correlating with clinical scales \citep{rezaii2019machine, bilgrami2022construct}. The majority of studies that compare these two groups of methods find syntactic methods more successful than LM-based methods both in differentiating between the groups\footnote{\citet{mitchell2015quantifying, iter2018automatic, corcoran2018prediction, just2020modeling, morgan2021natural, bilgrami2022construct, argolo2023burnishing}} and in predicting various clinical scales \citep{iter2018automatic, liebenthal2022linguistic, jeong2023exploring}, with very few exceptions \citep{rezaii2019machine}.

\begin{table}[h]
\resizebox{\textwidth}{!}{%
\begin{tabular}{llll}
\hline
\textbf{LM} & \textbf{Syntactic} & \textbf{Group Difference} & \textbf{Clinical Scales} \\ 
\hline
+ & + & \cite{bar2019semantic} & \textit{\cite{argolo2023burnishing}} \\
? & + & \begin{tabular}[c]{@{}l@{}}\cite{iter2018automatic} \textit{\cite{corcoran2018prediction}}; \\ \textit{\cite{morgan2021natural}}\end{tabular} & \cite{jeong2023exploring} \\
? & ? & \cite{tang2021natural} & \begin{tabular}[c]{@{}l@{}}\textit{\cite{rezaii2019machine}}; \\ \textit{\cite{bilgrami2022construct}}\end{tabular} \\
? & ! & \textit{\cite{rezaii2019machine}} &                                         \\
! & + & \begin{tabular}[c]{@{}l@{}}\cite{mitchell2015quantifying, just2020modeling}; \\ \textit{\cite{bilgrami2022construct, argolo2023burnishing}}\end{tabular}  & \begin{tabular}[c]{@{}l@{}}\cite{iter2018automatic}; \\ \cite{liebenthal2022linguistic}\end{tabular} \\
! & ! & \textit{\cite{haas2020linking}} & \begin{tabular}[c]{@{}l@{}}\textit{\cite{bedi2015automated}}; \\ \textit{\cite{corcoran2018prediction}}\end{tabular} \\ 
\hline
\end{tabular}}
\\
\captionsetup{width=\textwidth}
\caption[Comparison of language model-based and syntactic methods.]{\label{tab:comparison:LM-Synt} Comparison between language model-based (LM) and syntactic methods (Synt). 
\\ ``+'' indicates significant group difference or correlation for most metrics tested within the group. ``?'' indicates mixed results with some metrics showing significant results but not others. ``!'' indicates absence of significant differences in the metrics tested. The studies on clinical high risk populations are shown in italics.}
\end{table}

As shown in table \ref{tab:comparison:LM-Lex}, quite a few studies compare some LM-based to some lexical methods with similar results. While some find both groups efficient in predicting clinical scales \citep{vail2018toward}, many report neither of these approaches being efficient for CHR populations \citep{hitczenko2021understanding, argolo2023burnishing}. Once more, many find lexical methods more efficient than LM-based ones both in differentiating between the groups \citep{mitchell2015quantifying, just2019coherence, just2020modeling, aich2022towards} and in assessing symptom severity \citep{girard2022computational, hitczenko2021understanding}, again with some exceptions \citep{voppel2023semantic}.


\begin{table}[h]
\resizebox{\textwidth}{!}{%
\begin{tabular}{llll}
\hline
\textbf{LM} & \textbf{Lexical} & \textbf{Group Difference} & \textbf{Clinical Scales} \\ \hline
+ & + & & \cite{vail2018toward} \\
+ & ! & \cite{voppel2023semantic} & \\
? & + & \cite{just2019coherence} & \begin{tabular}[c]{@{}l@{}} \textit{\cite{rezaii2019machine}};\\ \cite{jeong2023exploring} \end{tabular} \\
! & + & \begin{tabular}[c]{@{}l@{}} \cite{mitchell2015quantifying, just2020modeling}; \\ \cite{aich2022towards}\end{tabular} & \cite{girard2022computational} \\ 
! & ! & \textit{\cite{hitczenko2021understanding, argolo2023burnishing}} & \textit{\cite{hitczenko2021understanding}} \\\hline
\end{tabular}}
\captionsetup{width=\textwidth}
\caption[Comparison of language model-based and lexical methods]{\label{tab:comparison:LM-Lex} Comparison between language model-based (LM) and lexical methods. 
\\ ``+'' indicates significant group difference or correlation for most metrics tested within the group. ``?'' indicates mixed results with some metrics showing significant results but not others. ``!'' indicates absence of significant differences in the metrics tested. The studies on clinical high risk populations are shown in italics.}
\end{table}

Table \ref{tab:comparison:Lex-Synt} compares lexical and syntactic methods with some studies successfully using both approaches \citep{mitchell2015quantifying, just2020modeling, jeong2023exploring} and some reporting no effect with either \citep{liang2022widespread}. Overall, neither approach can be deemed better, as some report lexical methods being more effective than syntactic \citep{gupta2018automated, rezaii2019machine}, while others observe the opposite trend \citep{schneider2023syntactic, argolo2023burnishing}.

\begin{table}[h]
\resizebox{\textwidth}{!}{%
\begin{tabular}{llll}
\hline
\textbf{Lexical} & \textbf{Syntactic} & \textbf{Group Difference} & \textbf{Clinical Scales} \\ \hline
+ & + &  \cite{mitchell2015quantifying, just2020modeling} & \cite{jeong2023exploring} \\
+ & ? & & \textit{\cite{rezaii2019machine}} \\
? & ! & \textit{\cite{gupta2018automated}} & \\
! & + & \cite{schneider2023syntactic}; \textit{\cite{argolo2023burnishing}} & \\
! & ! & \cite{liang2022widespread} &  \\\hline
\end{tabular}}
\captionsetup{width=\textwidth}
\caption[Comparison of lexical and syntactic methods.]{\label{tab:comparison:Lex-Synt} Comparison between lexical and syntactic methods. 
\\ ``+'' indicates significant group difference or correlation for most metrics tested within the group. ``?'' indicates mixed results with some metrics showing significant results but not others. ``!'' indicates absence of significant differences in the metrics tested. The studies on clinical high risk populations are shown in italics.}
\end{table}

Finally, Table \ref{tab:comparison:Graph-Synt-LM} compares graph methods to syntactic and LM-based ones. Most studies report graph-based methods being equally successful in differentiating between groups as syntactic methods \citep{spencer2021lower, morgan2021natural, nettekoven2023semantic}, though \citet{argolo2023burnishing} find better success with syntactic methods than graph-based ones in predicting symptom severity. As for LM-based methods, \citet{morgan2021natural} report limited success as compared to graph-based methods for finding group differences, while \citet{argolo2023burnishing} report the opposite pattern for predicting symptom severity.

\begin{table}[h]
\resizebox{\textwidth}{!}{%
\begin{tabular}{lllll}
\hline
\textbf{Graph} & \textbf{Synt} & \textbf{LM} & \textbf{Group Difference} & \textbf{Clinical Scales} \\ \hline
+ & + & & \textit{\cite{spencer2021lower, nettekoven2023semantic}} & \\
+ & + & ? & \textit{\cite{morgan2021natural}} & \\
? & + & + & & \textit{\cite{argolo2023burnishing}} \\ \hline
\end{tabular}}
\captionsetup{width=\textwidth}
\caption[Comparison of graph-based, syntactic, and language model-based methods.]{\label{tab:comparison:Graph-Synt-LM} Comparison between graph-based (Graph), syntactic (Synt), and language model-based methods (LM). 
\\ ``+'' indicates significant group difference or correlation for most metrics tested within the group. ``?'' indicates mixed results with some metrics showing significant results but not others. ``!'' indicates absence of significant differences in the metrics tested. The studies on clinical high risk populations are shown in italics.}
\end{table}

%----------------------------------------------------------------------------------------
%	SECTION 6
%----------------------------------------------------------------------------------------

\section{\textcolor{red}{Conclusion}}
\label{sec:review:summary}

All in all, despite a great variety of metrics proposed, no single metric seems to have strong evidence for being a robust indicator of psychotic alterations in speech especially with respect to cross-linguistic and cross-model reliability. 

% \textcolor{red}{which metrics are most commonly used?}

Coherence is by far the most frequently used NLP metric, yet the results are mixed. The same can be said of the moving window coherence, slope tangentiality, and word-based coherence metrics.  TTR and emotion words are the most popular metrics among lexical methods. Part-of-Speech frequencies, sentence length and count are the most frequently used syntactic metrics. Graph-based approaches utilize a similar set of metrics, which typically include size of LSC and LCC, as well as their randomness, and the number of nodes and edges. Additionally, many studies conducted on spoken discourse report disfluency features such as perseverations, hesitation pauses, and false-starts. Many studies report lower word counts as well as shorter sentences in affected populations.


% coherence 22
% average window coherence 10
% tangentiality (slope) 9
% SOC 5
% adjacent word similarity 5

% LD: TTR 8
% SA: emotion words 5
% # words 28

% POS 13
% determiners and wh-words
% adj adv
% verb noun
% sentence length 14
% sent # 8

% # nodes in LSC	11
% # nodes in LCC	8
% # nodes	7
% # edges	7
% LSCz	6
% LCCz	5
% "density
% 2E/N(N − 1)"	5
% diameter	5
% ASP	5

% preservation / repeats	 7
% false-start / restart	6
% hesitation pause 7

% \textcolor{red}{which metrics work the best?}

Overall, the results reported both for syntactic and lexical methods are more consistent than the ones for LM-based methods, which might be partially caused by the differences stemming from embedding model and preprocessing differences. Too few studies compare graph based methods to the other groups, yet the results of the graph-based studies are quite consistent within the group. Across methods, the results are considerably less consistent in clinical high risk populations both for the task of predicting psychosis conversion and for the task of differentiating CHR from controls, and CHR speech metrics were repeatedly reported as more similar to HC than FEP \citep{morgan2021natural, srivastava2022p473, nettekoven2023semantic}. It seems that negative symptoms are also more prevalent in the analyzed samples and can generally be approximated more robustly than positive symptoms. This could be because negative symptoms may occur before positive ones in the course of a psychotic disorder \citep{just2023validation}.

% Synt > LM
% Lex > LM
% Synt \& LM most popular
% Graph +- good

\textcolor{red}{The present work will include a cross-group comparison of the most frequently used and most successful metrics and perform several theoretical validation experiments for these metrics.}
% \textcolor{red}{which metrics are easiest to reproduce?}

% \textcolor{red}{which metrics will I test?}
